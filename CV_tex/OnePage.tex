\documentclass[english,a4paper,nologo,notitle]{europasscv}
\ecvname{Weipeng YAO}
\ecvaddress{Le Plessis-Robinson} \ecvtelephone{+33 6 43 65 42 49}
\ecvemail{yao.weipeng@polytechnique.edu}
\ecvhomepage{https://github.com/weipengyao}
\ecvdateofbirth{- 22/05/1990}
\ecvnationality{- Chinese}
\ecvgender{- Male}
% \ecvpicture[width=1.2in]{Photo_WeipengYAO}
% \ecvpicture[width=0.8in]{yaowp.jpg}
\ecvpictureleft
\usepackage{hyperref}

\definecolor{csc}{rgb}{0.0,0.0,1}

\hypersetup{
colorlinks,
urlcolor=csc,
    % pdfauthor={Weipeng YAO},
    % pdftitle={Weipeng YAO - Curriculum Vitae},
    % pdfsubject={Curriculum Vitae},
    % pdfkeywords={Curriculum Vitae, CV, Weipeng YAO, HEDLA}
}

\begin{document}
	\begin{europasscv}
	


	
	\newpage
	\ecvpersonalinfo
 
 \ecvsection{Summary}
 \ecvtitle{}{\small \rm  \color{black} I am currently a postdoc researcher at Ecole Polytechnique, CNRS, working in the field of laser-plasma interactions and fusion energy. 
 I have strong expertise in both numerical simulations in high-performance computing (HPC) systems and experimental skills in high-power laser facilities. I am good at collaboration and teamwork, and I have strong adaptability and flexibility.}
 


 \ecvsection{Work Experience}
	\ecvtitle{\color{black} 2019 -- Now}{\color{black} \small \bf Laboratoire pour l'Utilisation des Lasers Intenses (\href{https://luli.ip-paris.fr/chercheurs/equipes-de-recherche/sprint-sources-de-particules-rayonnement-intenses}{LULI})}
	% \ecvitem{Post-doc Researcher}{Advisor: \href{https://portail.polytechnique.edu/luli/en/dr-julien-fuchs}{Julien Fuchs} \& \href{https://sites.google.com/site/andreaciardihomepage/home}{Andrea Ciardi} }
    % \ecvitem{}{\rm \small \bf Investigation of plasma interacting with high-power lasers and strong magnetic fields:}
    \ecvitem{}{\rm  - open-source and collaborative plasma simulation codes, i.e., \textbf{\href{https://smileipic.github.io/Smilei/}{SMILEI}} and \textbf{\href{https://epochpic.github.io/}{EPOCH}};}
    \ecvitem{}{\rm - use the world's most powerful HPC systems, i.e., \textbf{\href{https://www.scinethpc.ca/niagara/}{Niagara}} and \textbf{\href{https://en.wikipedia.org/wiki/Sunway_TaihuLight}{Sunway TaihuLight}};}
    \ecvitem{}{\rm  - use the world's most powerful laser facilities, i.e., \textbf{\href{https://apollonlaserfacility.cnrs.fr/en/home/}{Apollon}} and \textbf{\href{https://www.clf.stfc.ac.uk/Pages/Vulcan-laser.aspx}{Vulcan}}.}
    \ecvitem{}{\bf \small All of them involve heavy data analysis \& visulization, see details in my {\href{https://scholar.google.com/citations?user=gzxsWFIAAAAJ&hl=en}{papers}}.} 
    
 \ecvsection{Skills}
	\ecvtitle{\color{black} \bf Data Analysis}{}
		\ecvitem{  \rm Proficient }{ \rm  Python, \LaTeX, Linux/Unix, HPC, Adobe illustrator, Fortran, Matlab}
		\ecvitem{  \rm Master }{  \rm Bash, C++,  VisIt, ParaView, OpenMP/MPI, HDF5}
		\ecvitem{ \rm Familiar with }{ \rm  Vim,  Inkscape}
				
	\ecvtitle{\color{black} \bf Code Projects}{}
	\ecvitem{\rm 2022 -- Now }{\small  \rm Open source, Particle-in-cell code with adaptive mesh refinement \textbf{\href{https://pharehub.github.io/}{PHARE}}, written in C++}	
        
        \ecvitem{\rm 2019 -- Now }{\small  \rm Open source, fully kinetic, massively parallel, Particle-in-cell code \textbf{\href{https://smileipic.github.io/Smilei/}{SMILEI}}, written in C++}
		\ecvitem{\rm 2019 -- Now }{\small  \rm  Resistive magneto-hydrodynamic code \textbf{GORGON}, written in Fortran}
		% \ecvitem{\rm 2019 - 2021 }{\small  \rm Proficient in particle module in 3D Ex-MHD code GORGON, written by Fortran.}
        \ecvitem{\rm 2020 -- 2021 }{\small  \rm Fully integrated particle physics Monte Carlo simulation package \textbf{\href{http://www.fluka.org/fluka.php}{FLUKA}}, written in Fortran}
		\ecvitem{\rm 2019 -- 2020 }{\small  \rm The radiation hydrodynamic code \textbf{MULTI}, written in C++.}
		% \ecvitem{\rm 2018 - 2019 }{\small  \rm Familiar with the HEDP Simulation using FLASH Code.}
		\ecvitem{\rm 2012 -- 2019 }{\small  \rm Open source, fully kinetic, massively parallel, Particle-in-cell code \textbf{\href{https://epochpic.github.io/}{EPOCH}}, written in Fortran.}
  
  \ecvtitle{\color{black} \bf Teaching}{\color{black} \rm \small  at Sorbonne University during academic year 2023-2024}
		\ecvitem{  \rm Master 1 }{ \rm  Numerical Tools in Physics}
		\ecvitem{  \rm Master 2 }{  \rm \href{https://github.com/weipengyao/Teaching_M2_2023}{Numerical Methods}}
  
 \ecvsection{Education}  
    
		\ecvitem{\rm 2015 -- 2019}{ \rm  Ph. D.: Plasma Physics, Peking University, Beijing, China (\bf TOP 2)} 	
            % \ecvitem{Thesis}{Kinetic study of relativistic jet and plasmas interaction in high energy astrophysics}

		\ecvitem{\rm 2012 -- 2015}{ \rm  Master of Science: Plasma Physics, China Academy of Engineering Physics, Beijing, China}
            % \ecvitem{Thesis}{Particle simulation research on monochromatic proton acceleration via ultra-short ultra-intense laser pulse and multi-component plasma interaction}
  
  		\ecvitem{\rm  2008 -- 2012}{  \rm  Bachelor of Science: Physics, Shanxi University, Taiyuan, China}

	                 
  	\end{europasscv}
\end{document}