\documentclass[english,a4paper,nologo,notitle]{europasscv}
\ecvname{Weipeng YAO}
\ecvaddress{3 Rue Bagno à Ripoli, 92350 Le Plessis-Robinson}
\ecvtelephone{+33 6 43 65 42 49}
\ecvemail{yao.weipeng@polytechnique.edu}
\ecvdateofbirth{- 22/05/1990}
\ecvnationality{- Chinese}
\ecvgender{- Male}
\ecvpicture[width=1.2in]{Photo_WeipengYAO}
% \ecvpicture[width=0.8in]{yaowp.jpg}
\ecvpictureleft
\usepackage{hyperref}

\definecolor{csc}{rgb}{0.0,0.0,1}

\hypersetup{
colorlinks,
urlcolor=csc,
    % pdfauthor={Weipeng YAO},
    % pdftitle={Weipeng YAO - Curriculum Vitae},
    % pdfsubject={Curriculum Vitae},
    % pdfkeywords={Curriculum Vitae, CV, Weipeng YAO, HEDLA}
}

\begin{document}
	\begin{europasscv}
	


	
	\newpage
	\ecvpersonalinfo

 	\ecvsection{Education}  
		\ecvtitle{\color{black} 09/2015 -- 07/2019}{\color{black} \small Ph. D.: Plasma Physics, School of Physics, Peking University, Beijing, China} 			
		\ecvtitle{\color{black} 09/2012 -- 07/2015}{ \color{black} \small Master of Science: Plasma Physics, Graduate School of China Academy of Engineering Physics, Beijing, China}
  		\ecvtitle{\color{black} 09/2008 -- 07/2012}{ \color{black} \small Bachelor of Science: Physics ({\it National Training Bases for Talented Students in Fundamental Sciences}), School of Physics , Shanxi University, Taiyuan, China}


\ecvsection{Research experience}
	\ecvtitle{\color{black} 10/2019 -- Now}{\color{black} \small LULI, École Polytechnique and LERMA, Sorbonne Université}
		\ecvitem{Post-doc Researcher}{Advisor: \href{https://portail.polytechnique.edu/luli/en/dr-julien-fuchs}{Julien Fuchs} \& \href{https://sites.google.com/site/andreaciardihomepage/home}{Andrea Ciardi} }
		\ecvitem{}{\rm \small Focuses on both theoretical and experimental investigation of laboratory and astrophysical plasma systems under magnetic fields, as well as laser-driven energitic particle sources.}

	\ecvtitle{\color{black} 09/2015 -- 07/2019}{\color{black} \small Center for Applied Physics and Technology, Peking University, Beijing, China}
		\ecvitem{Ph.D. Candidate}{Advisor: Xian-tu He \& \href{https://faculty.pku.edu.cn/qiaobin/en/index.htm}{Bin Qiao} }
		\ecvitem{}{\rm \small Focuses on the theoretical study of astrophysical relativistic jet transport in plasma, including the competing of different kinetic instabilities, magnetic reconnection, particle acceleration via collisionless shock and $\gamma$-ray radiation via quantum electrodynamics (QED) effects.}
		\ecvitem{Thesis}{Kinetic study of relativistic jet and plasmas interaction in high energy astrophysics}

	% \ecvtitle{\color{black} 01/12/2017 -- 29/12/2017}{\color{black} \small Shanghai Institute of Laser Plasma, China Academy of Engineering Physics, Shanghai, China}
	% 	\ecvitem{Experimental participant}{Advisor: Prof. Wenbin Pei}
	% 	\ecvitem{}{\rm \small Focuses on the asymmetrical laser-driven magnetic reconnection (MR) using proton radiography at SG-II-upgrade laser facility.}

	\ecvtitle{\color{black} 09/2012 -- 07/2015}{\color{black} \small Graduate School of China Academy of Engineering Physics, Beijing, China}
		\ecvitem{Master Candidate}{Advisor: Prof. Baiwen Li}
		\ecvitem{}{\rm \small Focuses on the theoretical investigation of high-energy and high-quality proton beam generation based on laser plasma interaction.}
		\ecvitem{Thesis}{Particle simulation research on monochromatic proton acceleration via ultra-short ultra-intense laser pulse and multi-component plasma interaction}
		
		
\ecvsection{Recent Research Interests}
	
	\ecvitem{}{\color{black}  \rm With the development of laser technology, it will be more and more convenient to create plasmas under high-energy-density (HED) regime in laboratory. My research interest lies in this new, physically rich and exciting area at the intersection of traditional plasma physics, high-energy astrophysics and HED plasma physics. 
    Specifically, I am interested in the origin of cosmic rays, which is believed to be accelerated from collisonless shocks, especially when magnetic field effects are non-negligible. In addition, I am also interested in the laser-driven ultra-bright neutron source, which opens up the laboratory investigations of the nucleosynthesis of heavy elements, by using ultra-intense short-pulse laser facilities, e.g., Apollon. Besides, the advanced numerical modeling of related systems with both kinetic Particle-in-Cell (PIC) codes and extended magnetohydrodynamic (MHD) codes are also my research interests. 
    \bf For an up to date and exhaustive list of articles see my profile on {\href{https://scholar.google.com/citations?hl=en&user=gzxsWFIAAAAJ}{\underline{google scholar}}}
    }

	                 
  	\end{europasscv}
\end{document}